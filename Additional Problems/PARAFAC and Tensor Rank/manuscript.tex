\documentclass[a4paper,10pt]{article}

\usepackage{cite}
\usepackage{graphicx}
\usepackage{fancyhdr}
\usepackage{amstext,amsmath,amssymb}
\usepackage[shortlabels]{enumitem}

\setlength{\oddsidemargin}{0cm} %
\setlength{\evensidemargin}{0cm} %
\setlength{\topmargin}{0cm} %
\setlength{\textwidth}{16cm} %
\setlength{\textheight}{22.5cm} %

\pagestyle{fancy}
\newcommand{\assunto}{PARAFAC and Tensor Rank}

\sloppy

\begin{document}

\thispagestyle{empty}

\begin{center}
  
    \includegraphics[scale=0.10]{figs/icon.png}
    
    \LARGE{Universidade Federal do Ceará}
    
    \LARGE{Centro de Tecnologia}
    
    \LARGE{Departamento de Engenharia de Teleinformática}
    
    \LARGE{Engenharia de Teleinformática}
    
    \vspace{180pt}
      
    \LARGE{Multilinear Algebra}
      
    \LARGE{PARAFAC and Tensor Rank}
      
    \vspace{100pt}
    
\end{center}

\vspace{25pt}

\begin{flushleft}
	\begin{tabbing}
		Student \qquad Kenneth Brenner dos Anjos Benício – 519189\\
	   \qquad\qquad\qquad\= \\
		Professor\> Andre Lima Ferrer de Almeida \\
		Course \> Multilinear Algebra - TIP8419\\
	\end{tabbing}
\end{flushleft}

\vspace{25pt}

\begin{center}
    Fortaleza, 2021
\end{center}
\thispagestyle{empty}

\newpage

\thispagestyle{empty}

\begin{enumerate}
\renewcommand{\labelenumi}{{\Large\bfseries\arabic{enumi}.}}
   
    \item We know that if we have the tensor defined as
    
        \begin{align}
            \mathcal{X} = \boldsymbol{a}_{1} \circ \boldsymbol{b}_{1} \circ \boldsymbol{c}_{1} + \boldsymbol{a}_{2} \circ \boldsymbol{b}_{2} \circ \boldsymbol{c}_{2},
        \end{align}

        then if we have $\boldsymbol{b}_{1} = \boldsymbol{b}_{2}$ and $\boldsymbol{c}_{1} = \boldsymbol{c}_{2}$ we can guarantee that $\mathcal{X}$ is rank one. We can begin this proof by using the associativity property of the outer product to write tensor $\mathcal{X}$ as

        \begin{align}
            \mathcal{X} &= \boldsymbol{a}_{1} \circ \boldsymbol{b}_{1} \circ \boldsymbol{c}_{1} + \boldsymbol{a}_{2} \circ \boldsymbol{b}_{2} \circ \boldsymbol{c}_{2}, \\
            \mathcal{X} &= \boldsymbol{a}_{1} \circ \boldsymbol{b}_{1} \circ \boldsymbol{c}_{1} + \boldsymbol{a}_{2} \circ \boldsymbol{b}_{1} \circ \boldsymbol{c}_{1}, \\
            \mathcal{X} &= (\boldsymbol{a}_{1} + \boldsymbol{a}_{2}) \circ \boldsymbol{b}_{1} \circ \boldsymbol{c}_{1}.
        \end{align}
        
        By inspection of the expression above we can see that the elements that compose vectors $\boldsymbol{a}_{1}$ and $\boldsymbol{a}_{2}$
        acts as weights in a linear combination of the matrix $\boldsymbol{b}_{1} \circ \boldsymbol{c}_{1}$. Thus, if we assume that the tensor is not rank one then we should at least one of the vectors equals to zero so we can obtain a sum of linearly independent terms that leads to a tensor rank greater than one. 
        However, this does not makes sense because if we have one of these vectors equals to zero then  we would not have a sum at all. Thus, by contradiction, we know that in the proposed scenario the vectors $\boldsymbol{a}_{1}$ and $\boldsymbol{a}_{2}$ must be collinear meaning that the tensor is indeed rank one.
        In a similar fashion we can obtain a conclusion for the case
        where we have $\boldsymbol{b}_{1} \neq \boldsymbol{b}_{2}$ and $\boldsymbol{c}_{1} = \boldsymbol{c}_{2}$ by writting the tensor as

        \begin{align}
            \mathcal{X} &= \boldsymbol{a}_{1} \circ \boldsymbol{b}_{1} \circ \boldsymbol{c}_{1} + \boldsymbol{a}_{2} \circ \boldsymbol{b}_{2} \circ \boldsymbol{c}_{2}, \\
            \mathcal{X} &= \boldsymbol{a}_{1} \circ \boldsymbol{b}_{1} \circ \boldsymbol{c}_{1} + \boldsymbol{a}_{2} \circ \boldsymbol{b}_{2} \circ \boldsymbol{c}_{1}, \\
            \mathcal{X} &= (\boldsymbol{a}_{1} \circ \boldsymbol{b}_{1} + \boldsymbol{a}_{2} \circ \boldsymbol{b}_{2}) \circ \boldsymbol{c}_{1}.
        \end{align}

        By inspecting the expression above by the same procedure as before we can once again reach the same conclusion that the only way to the tensor $\mathcal{X}$ to have a rank greater than one in this 
        scenario is if one of the vectors is zero, but that would be contradictory. Thus, we can guarantee once more that the tensor $\mathcal{X}$ will be rank one in this case.
  
    \item
    
    \item

        \begin{enumerate}
                
            \item

            \item

            \item

            \item

        \end{enumerate}

    \item 

        \begin{enumerate}
            
            \item

            \item

            \item 

        \end{enumerate}

\end{enumerate}

%\bibliographystyle{ieeetr}
%\bibliography{bibliography.bib}

\end{document}

